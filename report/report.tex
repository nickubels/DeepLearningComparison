\documentclass{article}
\usepackage{bnaic}


%% if your are not using LaTeX2e use instead
%% \documentstyle[bnaic]{article}

%% begin document with title, author and affiliations

\title{\textbf{\huge BNAIC\\ Instructions to Authors}%
\footnote{In case of an extended abstract refer to the original paper in a footnote such as ``The full paper has been published in \emph{Proceedings of the International Joint Conference on Artificial Intelligence}, pages 13--20, 2013.'' Also, please keep the title and authors exactly the same as the original.}%
}
\author{First author \affila \and
    Second author \affilb \and
    Third author \affila}
\date{\affila\ \textit{University of Franeker, P.O.Box 1234 1879AE Franeker}\\
    \affilb\ \textit{The Company Ltd. P.O.Box 4321 Antwerp}}

\pagestyle{empty}

\begin{document}
\ttl
\thispagestyle{empty}


\begin{abstract}
\noindent
This is the abstract of my paper. Please start the first paragraph of your abstract with a \verb+\noindent+ command.
This is the abstract of my paper. Please start the first paragraph of your abstract with a \verb+\noindent+ command.
This is the abstract of my paper. Please start the first paragraph of your abstract with a \verb+\noindent+ command.
\end{abstract}


\section{The bnaic Package}

The \verb+bnaic.sty+ file is a package that is to be used together with
the standard \verb+article+ document class. Please adhere strictly to the instructions of this document. The bnaic style file uses the standard \verb+times+ package and the \verb+geometry+ package, which is included in the bnaic package; please do not change it! 

\section{The Title}

The title of the article has to appear in bold and the \verb+\huge+ keyword has to be used to set the correct font size. For
the authors and their affiliations, three cases are distinguished:

\begin{itemize}
\item One author: define the author with \verb+\author+ and the affiliation
   with \verb+\date+.
\item Multiple authors, all with the same affiliation: define the authors with
   \verb+\author+, separated by \verb+\and+, and the affiliation with
   \verb+\date+.
\item Multiple authors, multiple affiliations: define the
   authors with \verb+\author+. Put after each name a letter for the
   affiliation, generated by \verb+\affila+, \verb+\affilb+, etc. On the
   next lines: one affiliation per line, each preceded by the appropriate letter
   generated by \verb+\affila+, \verb+\affilb+. See the title of this document.
\end{itemize}

Note that affiliations have to appear in italics.

\section{The abstract}

Put the abstract before the first section with the \verb+abstract+
environment. Please start the first paragraph of your abstract with a \verb+\noindent+ command.


\section{Sections and paragraphs}

  Use the usual \verb+\section+, \verb+\subsection+, and
  \verb+\subsubsection+ command for formatting section heads.

  Do not leave extra space between paragraphs.


  \subsection{A Subsection Heading}


  \subsubsection{A Subsubsection Heading}


  \section{Making References}

  Make references in the running text with the \verb+\cite+
  command \cite{dijkstra68}. Multiple references go like this
  \cite{charniak85,steels98}.


\bibliographystyle{plain}
\bibliography{mybibfile}



\end{document}








